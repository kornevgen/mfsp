\documentclass[12pt, a4paper]{article}
\usepackage[utf8]{inputenx}
\usepackage[russian]{babel}

\usepackage{amsmath}

\usepackage[pgf]{dot2texi}
\usepackage{tikz}
\usetikzlibrary{shapes, arrows}

\title{Домашняя работа №2}
\date{}
\author{}

\begin{document}

\maketitle

Множество переменных $V = \{ x, y_1, y_2, y_3, z \}$ состоит из одной входной, трех промежуточных и одной выходной переменных. Доменом всех переменных является множество всех целых чисел. На рисунке представлена блок-схема программы над множеством переменных $V$, вычисляющая целую часть кубического корня. Доказать частичную корректность программы относительно входного предиката $\varphi(x)~\equiv~(x \geq 0)$ и выходного предиката $\psi(x,~z)~\equiv~( z^3 \leq x < (z + 1)^3 )$ при помощи методов Флойда.

\hspace{2cm}

\begin{dot2tex}[options=-traw]
	digraph G{
		d2tgraphstyle="scale=0.5";

		/* nodes by levels */
		node[shape=rectangle, height=1];
		START[style=rounded, width=4, texlbl="$\begin{matrix}START:\\ (y_1,~y_2,~y_3) \leftarrow (0,~0,~1)\\\end{matrix}$"];
		ASSIGN1[width=3, texlbl="$y_2 \leftarrow y_2 + y_3$"];
        TEST[style=rounded, width=2, texlbl="$y_2 > x$"];
		ASSIGN2[width=3, texlbl="$y_1 \leftarrow y_1 + 1$"];
		ASSIGN3[width=3, texlbl="$y_3 \leftarrow y_3 + 6 \cdot y_1$"];
		HALT[style=rounded, width=3, texlbl="$\begin{matrix}HALT:\\ z \leftarrow y_1\end{matrix}$"];

        /* edges */
		node [shape=point, width=0, label=""];
		START -> p2 [arrowhead=none]; p2 -> ASSIGN1 [weight=8];
		{ rank=same; p1 -> p2; }
		p1 -> p5 [weight=8, arrowhead=none];
        ASSIGN1 -> TEST[weight=8];
		{ rank=same; p3 -> TEST [label="F", arrowhead=none]; TEST -> p4 [label="T", arrowhead=none]; }
		p3 -> ASSIGN2 [weight=8];
		p4 -> HALT [weight=8];
		{ rank=same; ASSIGN2; HALT; }
        ASSIGN2 -> ASSIGN3 [weight=8];
		ASSIGN3 -> p6 [weight=8, arrowhead=none];
		{ rank=same; p5 -> p6 [arrowhead=none]; }
	}
\end{dot2tex}

\end{document}

