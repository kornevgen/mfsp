\documentclass[hyperref={unicode=true}]{beamer}
\usepackage[utf8]{inputenx}
\usepackage[russian]{babel}

\usepackage{multicol}

\usepackage[pgf]{dot2texi}
\usepackage{tikz}
\usetikzlibrary{shapes, arrows}

\usepackage{listings}
\usepackage{graphicx}

\usepackage{comment}

\title{Лекция 6. Леммы, ghost-блок-схемы, auto-active verification}
\author{}
\date{}

\usetheme{Warsaw}

\AtBeginSection[] {
	\begin{frame}{Содержание}
		\tableofcontents[currentsection]
	\end{frame}
}
%\overfullrule=5pt

\begin{document}
	\begin{frame}{}
		\titlepage
	\end{frame}

    \begin{frame}{Цель лекции}
    Научиться еще эффективнее использовать SMT-солверы
    для дедуктивной верификации.
    \end{frame}

    \section{Леммы}

    \begin{frame}{Добавление термов}
    \begin{itemize}
    \item
    Продолжаем доказывать полную корректность блок-схемы,
    вычисляющей квадратный корень. Часть условий верификации
    пока не доказаны. Продолжаем их доказательство.
    \item
    Анализ триггеров показывает, что солвер \emph{может}
    не доказывать условия верификации, т.к. нечем инстанцировать
    квантор.
    \item
    Какой нужен терм (чтобы получить нужное противоречие)?
    Куда можно добавить этот терм?
    \end{itemize}
    \end{frame}

    \begin{frame}{Место добавления}
    \begin{itemize}
    \item
    Получается, нам нужно новое утверждение, которое даст и триггер,
    и противоречие. Куда добавляем?
    \item
    в предусловие или постусловие нельзя, это часть требований
    \item
    в индуктивное утверждение? К чему это приведет?
    \item
    в оценочную функцию? К чему это приведет?
    \item
    добавить новые точки сечения с нужными индуктивными утверждениями
    и оценочными функциями? К чему это приведет?
    \end{itemize}
    \end{frame}

    \begin{frame}{Лемма}
    \begin{itemize}
    \item
    Лемма -- это истинное утверждение, которое включается в посылку
    условий верификации.
    \item
    Лемму нужно тоже доказать.
    \item
    В \textsl{Why3} лемму надо оформить нужным образом и написать
    до целей -- условий верификации.
    \item
    Посмотрите в \textsl{Why3IDE}, как лемма добавляется в посылку
    условия верификации.
    \end{itemize}
    \end{frame}

    \begin{frame}{Лемма в примере}
    \begin{itemize}
    \item
    Составьте лемму, которая даст нужный терм для недоказанного
    условия верификации для пути START-A.
    \item
    Могут ли солверы доказать эту лемму?
    \item
    Могут ли солверы теперь доказать условие верификации?
    \end{itemize}
    \end{frame}

    \begin{frame}{Порядок лемм}
    \begin{itemize}
    \item
    Одной этой леммы недостаточно. Нужны дополнительные леммы.
    \item
    Важен ли порядок лемм? Важен! Проведите соответствующий
    эксперимент с \textsl{Why3IDE}.
    \item
    Итак, для доказательства леммы используются леммы, которые
    расположены до нее. Это нужно использовать для доказательства лемм.
    \item
    Добавьте нужные леммы для доказательств в нашем примере.
    \end{itemize}
    \end{frame}

    \begin{frame}{Индукция и солверы}
    \begin{itemize}
    \item
    Одна из лемм не доказывается (<<для любого неотрицательного
            числа существует целочисленный квадратный корень>>). Почему?
    \item
    Для доказательства этой леммы нужен метод математической
    индукции. Солвер не может автоматически применять этот метод
    (он не сводится к инстанцированию кванторов). Как доказать лемму
    по индукции?
    \end{itemize}
    \end{frame}

    \begin{frame}{Леммы и пруверы}
    \begin{itemize}
    \item
    Леммы бывают такими сложными (в них можно перенести всю сложность
    из условий верификации), что доказывать их из других лемм или сложно
    (сложно выписать нужные шаги доказательства леммы, если вообще
    возможно), или непредсказуемо (добавление лемм может привести к тому,
    что ранее доказывавшиеся леммы перестанут доказываться из-за
    возникновения новых лишних инстансов кванторов).
    \item
    Поэтому один из выходов -- доказательство лемм при помощи пруверов
    (мы вручную указываем путь доказательства в прувере).
    \item
    Мы посмотрим, как все же можно как можно больше доказать при помощи
    солверов.
    \end{itemize}
    \end{frame}

    \section{Ghost-блок-схемы}

    \begin{frame}{Вставка блок-схем}
    \begin{itemize}
    \item
    Почему бы не вставить вызов дополнительной блок-схемы
    в нашу блок-схему? Эта блок-схема будет иметь спецификацию,
    ее постусловие будет вставлено в условие верификации
    базовых путей, на которых стоит этот вызов.
    \item
    Нам важно постусловие -- оно будет вставлено в условие
    верификации. Это постусловие должно быть тем утверждением,
    которое необходимо для доказательства условия верификации.
    \end{itemize}
    \end{frame}

    \begin{frame}{Методы Флойда как метод доказательства утверждения}
    \begin{itemize}
    \item
    Само постусловие тоже надо доказать (как и лемму).
    Дополнительная блок-схема будет что-то делать. Главное --
    что доказательство полной корректности этой блок-схемы
    относительно нужной спецификации методами Флойда
    будет доказательством утверждения-постусловия.
    \item
    Можно использовать всё, что может быть в блок-схемах:
    циклы, рекурсию. Получим... доказательство методом
    математической индукции!
    \item
    Эта идея называется auto-active verification.
    \end{itemize}
    \end{frame}

    \begin{frame}{Много ручной работы}
    \begin{itemize}
    \item
    Предложите нужные блок-схемы и спецификации для
    завершения доказательства полной корректности
    примера с квадратным корнем.
    \item
    Докажите полную корректность всех блок-схем
    <<на бумажке>>.
    \item
    Теперь надо доказать полную корректность в Why3.
    Надо выписать много условий верификации. Нам нужен
    инструмент автоматизации их построения.
    \end{itemize}
    \end{frame}

\end{document}

