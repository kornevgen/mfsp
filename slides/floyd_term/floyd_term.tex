\documentclass[hyperref={unicode=true}]{beamer}
\usepackage[utf8]{inputenx}
\usepackage[russian]{babel}

\usepackage{multicol}

\usepackage[pgf]{dot2texi}
\usepackage{tikz}
\usetikzlibrary{shapes, arrows}

\usepackage{listings}
\usepackage{graphicx}

\usepackage{comment}

\title{Лекция 3. Метод фундированных множеств Флойда}
\author{}
\date{}

\usetheme{Warsaw}

\AtBeginSection[] {
	\begin{frame}{Содержание}
		\tableofcontents[currentsection]
	\end{frame}
}
%\overfullrule=5pt

\begin{document}
	\begin{frame}{}
		\titlepage
	\end{frame}

    \begin{frame}{Цель лекции}
    Определить метод доказательства завершимости.
    \end{frame}

    \section{Доказательство на примере}

	\begin{frame}[fragile]{Пример для доказательства}
	\begin{multicols}{2}

	\huge
	\begin{dot2tex}[options=-traw]
	digraph G{
		d2tgraphstyle="scale=0.4, transform shape";

		/* nodes by levels */
		node[shape=rectangle, height=1];
		START[style=rounded, width=2, texlbl="$\begin{matrix}START:\\ y \leftarrow 0\end{matrix}$"];
		JOIN;
        COND[style=rounded, width=2, label="$y = x$"];
		INCR[width=2, texlbl="$y \leftarrow y + 1$"];
        HALT[style=rounded, width=2, texlbl="$\begin{matrix}HALT:\\  z \leftarrow y\end{matrix}$"];

		/* edges */
		node [shape=point, width=0, label=""];
		START -> JOIN [arrowhead=none]; JOIN -> COND [weight=8];
		{ rank=same; p1 -> JOIN; }
		p1 -> p5 [weight=8, arrowhead=none];
		{ rank=same; p3 -> COND [label="F", arrowhead=none]; COND -> p4 [label="T", arrowhead=none]; }
		p3 -> INCR [weight=8];
		p4 -> HALT [weight=8];
		{ rank=same; INCR; HALT; }
		INCR -> p6 [weight=8, arrowhead=none];
		{ rank=same; p5 -> p6 [arrowhead=none]; }
	}
	\end{dot2tex}

	\normalsize

    $\begin{matrix}
    D_x = \mathbb{Z}\\
    D_y = \mathbb{Z}\\
    D_z = \mathbb{Z}\\
    \varphi(x) \geq 0\\
    \end{matrix}$

    Доказать, что блок-схема завершается при всех значениях входных переменных из указанного предусловия. Метод доказательства должен быть <<автоматизируемым>>.
	\end{multicols}

	\end{frame}


    \begin{frame}{Поиск доказательства}
    Надо доказать, что все вычисления при входных переменных таких, что $\varphi(x)$, завершаются, т.е. достигают связки перед оператором HALT, т.е. что достигают конфигурации перед TEST, в которой истинен предикат в TEST.

    Какие есть известные техники доказательства достижимости? Графы (из чего? из конфигураций). Из одной вершины есть дуга в другую вершину, если есть вычисление с этой парой конфигураций подряд. Предусловие дает некоторое подмножество вершин графа. Есть подмножество вершин - конфигураций перед TEST, в котороых истинен предикат в TEST. Надо доказать, что из каждой вершины одного множества существует путь в некоторую вершину второго множества. На графах применимы техники динамического программирования. Но здесь они не применимы из-за возможной бесконечности исходного множества и бесконечности графа.
    \end{frame}

    \begin{frame}{Поиск доказательства}
    Где еще встречалась достижимость в бесконечном случае? В принципе индукции. Там нужно сводить произвольные данные по переходам к базе. Если такая сводимость есть, то вместо доказательства бесконечного множества утверждений рассматривается доказательство конечного множества утверждений (про индуктивный переход и про базу). Сводимость --- то же, что и достижимость.

    Итак, надо доказать возможность проведения индукции по путям из START в HALT. Мы пользовались этой индукцией при доказательстве частичной корректности - она предполагала завершаемость. Теперь надо обосновать, что индукцию можно было проводить.
    \end{frame}

    \begin{frame}{Поиск доказательства}
    Весь вопрос в том, будет ли завершаться цепочка конфигураций с меткой оператора TEST в каждом вычислении, где выполнено $\varphi(x)$. Если при всех соответствующих значениях $x$ можно сопоставить этой цепочке убывающую последовательность натуральных чисел, то возможность индукции по путям (т.е. завершаемость блок-схемы) будет доказана, т.е. не существует бесконечно убывающей последовательности натуральных чисел.

    Попробуем добавить промежуточную переменную $u$, ее домен - $\{0,1,2,...\}$, не влияющую на вычисление. По ходу вычисления эта переменная должна уменьшаться. Получится, что вычисление не может быть бесконечным, так как иначе переменная выйдет за границы домена.
    \end{frame}

  	\begin{frame}[fragile]{Добавление убывающей переменной}
	\huge
	\begin{dot2tex}[options=-traw]
	digraph G{
		d2tgraphstyle="scale=0.4, transform shape";

		/* nodes by levels */
		node[shape=rectangle, height=1];
		START[style=rounded, width=3, texlbl="$\begin{matrix}START:\\ (y, u) \leftarrow (0, |x|)\end{matrix}$"];
		JOIN;
        COND[style=rounded, width=2, label="$y = x$"];
		INCR[width=5, texlbl="$(y, u) \leftarrow (y + 1, |x - (y + 1)|)$"];
        HALT[style=rounded, width=2, texlbl="$\begin{matrix}HALT:\\  z \leftarrow y\end{matrix}$"];

		/* edges */
		node [shape=point, width=0, label=""];
		START -> JOIN [arrowhead=none]; JOIN -> COND [weight=8];
		{ rank=same; p1 -> JOIN; }
		p1 -> p5 [weight=8, arrowhead=none];
		{ rank=same; p3 -> COND [label="F", arrowhead=none]; COND -> p4 [label="T", arrowhead=none]; }
		p3 -> INCR [weight=8];
		p4 -> HALT [weight=8];
		{ rank=same; INCR; HALT; }
		INCR -> p6 [weight=8, arrowhead=none];
		{ rank=same; p5 -> p6 [arrowhead=none]; }
	}
	\end{dot2tex}
	\end{frame}

    \begin{frame}{Доказательство примера}
    Попробуем доказать, что переменная $u$ уменьшается на каждой итерации цикла. То есть мы хотим доказать, что на всех вычислениях из $x$ таких, что $x \geq 0$, в любых соседних конфигурациях связки между JOIN и TEST $u_{n+1} < u_n$. Для этого предварительно показываем по индукции, что каждый раз на этой связке выполнено индуктивное утверждение $x \geq y \land u = |x-y|$ (так же, как в предыдущей лекции).

    \emph{База индукции}. Путь из псевдосвязки перед START в связку перед TEST: $u_0 = |x|$ (из START). Получается: $\forall x \in \mathbb{Z} \cdot x \geq 0 \Rightarrow x \geq 0 \land |x| = |x - 0|$. Доказано.

    \emph{Индуктивный переход}. Путь из связки перед TEST в связку перед TEST. Надо доказать, что $\forall x, y, u \in \mathbb{Z} u \geq 0 \Rightarrow x \geq 0 \land y \neq x \land x \geq y \Rightarrow x \geq (y + 1) \land |x - (y + 1)| = |x - (y + 1)|$. Это очевидно истинно. Доказано.

    Значит, на всех вычислениях на связке перед оператором TEST выполнено утверждение $x \geq y \land u = |x - y|$.
    \end{frame}

    \begin{frame}{Доказательство примера}
    Попробуем теперь доказать, что на каждом базовом пути из связки перед TEST в эту же связку переменная $u$ уменьшается.

    Пусть $u = |x - y|$ выполнено вначале пути, $u' = |x - (y + 1)|$ выполнено в конце пути, причем имеется предикат пути $y \neq x$. Тогда надо доказать, что $\forall x \in \mathbb{Z}, y \in \mathbb{Z}, u \in \mathbb{Z} u \geq 0 \Rightarrow x \geq 0 \land y \neq x \land x \geq y \land u = |x - y| \Rightarrow |x - (y + 1)| < |x - y|$. Оно истинно. Доказано.

    Значит, не может быть бесконечного вычисления, т.к. иначе переменная $u$ выйдет за свой домен.
    \end{frame}

    \begin{frame}{Упрощение доказательства}
    \begin{itemize}
    \item Можно не добавлять $u$ во все операторы START и ASSIGN, а ввести функцию от конфигурации, дающую те же значения переменной $u$.
    \item Можно ввести индуктивное утверждение и доказывать убывание $u$ из этого индуктивного утверждения.
    \item Можно использовать другое множество значений переменной $u$. Главное - чтобы в нем не было бесконечно убывающей последовательности значений.
    \item Можно вместо этого домена переменной $u$ рассматривать надмножество этого домена. Это упростит выкладки.
    \end{itemize}
    \end{frame}

	\section{Метод фундированных множеств}

    \begin{frame}{Предварительные определения}
	\emph{Отношение строгого частичного порядка} -- это бинарное отношение $\prec$ на некотором множестве $W$, обладающее следующими свойствами:
    \begin{enumerate}
    \item антирефлексивность: $\forall x \in W \cdot \neg (x \prec x)$.
    \item транзитивность: $\forall x, y, z \in W \cdot x \prec y \land y \prec z \Rightarrow x \prec z$.
    \end{enumerate}

    \emph{Фундированное множество} -- множество, снабженное отношением строгого частичного порядка, в котором не существует бесконечно убывающей последовательности элементов.
    \end{frame}

	\begin{frame}{Метод фундированных множеств}

    \begin{block}{Шаг 1}
	Выбор множества т.с. (все циклические пути имеют т.с.) и фундированного множества $(W,~\prec)$.
	\end{block}
	\begin{block}{Шаг 2}
	Выбор индуктивного утверждения для каждой т.с., выписывание условий верификации для каждого
	базового пути между точками сечения и псевдосвязкой у START.
	\end{block}
	\begin{block}{Шаг 3}
	Выбор оценочной функции для каждой точки сечения ($u_A~:~D_{\bar{x}}~\times~D_{\bar{y}}~\rightarrow~W'$, $W \subseteq W'$).
	\end{block}
	\end{frame}

	\begin{frame}{Метод фундированных множеств (продолжение)}

	\begin{block}{Шаг 4}
	Выписывание условия корректности оценочной функции для каждой точки сечения:
	$\forall \bar{x} \in D_{\bar{x}} ~\forall \bar{y} \in D_{\bar{y}} ~\cdot~
	\varphi(\bar{x}) \land p_A(\bar{x},~\bar{y}) \Rightarrow u_A(\bar{x},~\bar{y}) \in W$.
	\end{block}
	\begin{block}{Шаг 5}
	Выписывание условия завершимости для каждого базового пути между точками сечения (из А в В):
	$\forall \bar{x} \in D_{\bar{x}} ~ \forall \bar{y} \in D_{\bar{y}} ~\cdot~
	\varphi(\bar{x}) \land p_A(\bar{x},~\bar{y})~\land~R_\alpha(\bar{x},~\bar{y}) \Rightarrow
	u_B(\bar{x},~r_\alpha(\bar{x},~\bar{y})) ~\prec~ u_A(\bar{x},~\bar{y})$.
	\end{block}
	\end{frame}

	\begin{frame}{Корректность метода фундированных множеств}

	\begin{block}{Теорема}
	Дана блок-схема $P$, спецификация $(\varphi,~\psi)$. Если все составленные условия верификации, корректности и завершимости истинны, то $\langle\varphi\rangle~P~\langle T \rangle$, т.е. блок-схема завершима.
	\end{block}

    Схема доказательства: по индукции доказать выполнение индуктивных утверждений в точках сечения, из фундированности $W$ сделать вывод об отсутствии бесконечных вычислений.
	\end{frame}

	\begin{frame}{Примеры фундированных множеств}
	\begin{block}{Натуральные числа}
	$W~\equiv~\{0,~1,~2,~\ldots\}$ -- множество целых неотрицательных чисел

	$x~\prec~y~\equiv~x~<~y$ -- с естественным порядком на нем
	\end{block}
	\begin{block}{Кортежи}
	$W~\equiv~W_1~\times~W_2$ -- пара двух фундированных множеств $(W_1,~\prec_1)$ и $(W_2,~\prec_2)$.

	$(x_1,~x_2)~\prec~(y_1,~y_2) ~\equiv~ x_1~\prec_1~y_1~\lor~x_1~=~y_1 \land x_2~\prec_2~y_2$ -- лексикографический порядок.
	\end{block}
	\end{frame}

\end{document}

