\documentclass[hyperref={unicode=true}]{beamer}
\usepackage[utf8]{inputenx}
\usepackage[russian]{babel}

\usepackage{multicol}

\usepackage[pgf]{dot2texi}
\usepackage{tikz}
\usetikzlibrary{shapes, arrows}

\usepackage{listings}
\usepackage{graphicx}

\usepackage{comment}

\title{Лекция 9. Моделирование массивов}
\author{}
\date{}

\usetheme{Warsaw}

\AtBeginSection[] {
	\begin{frame}{Содержание}
		\tableofcontents[currentsection]
	\end{frame}
}
%\overfullrule=5pt

\begin{document}
	\begin{frame}{}
		\titlepage
	\end{frame}

    \begin{frame}{Цель лекции}
    Построить аксиоматику для массивов.
    \end{frame}

    \begin{frame}{Постановка задачи}
    \begin{itemize}
    \item
    Назовем последовательность значений одного типа
    логическим массивом. Домен массивов -- это множество
    всех логических массивов.
    \item
    Элементы логического массива проиндексированы, начиная с 0.
    \item
    У переменной с доменом массивов можно читать элемент,
    изменять элемент. При изменении элемента переменная получает
    новое значение. При создании массива задается его размер
    и начальное значение (одно) всех элементов.
    \item
    Нужно добавить домен массивов в блок-схемы.
    Для этого нужно придумать типы-символы,
    функциональные символы, предикатные символы и аксиомы.
    Придумать операторы CALL для работы с массивами.
    \end{itemize}
    \end{frame}

    \begin{frame}{Символы}
    \begin{itemize}
    \item \texttt{type array 't} -- логический массив
    \item \texttt{type arrayVar 't = \{ mutable items: array 't\}}
    -- тип переменных в блок-схеме с доменом массивов
    \item \texttt{function length (a: array 't)} -- длина массива,
    это функциональный символ, значит можем его использовать только
    в спецификациях
    \item \texttt{predicate valid\_index (a: array 't) (i: int) =
        0 <= i < length a}
    \item \texttt{function get (a: array 't) (i: int)} -- получение
    элемента с индексом \texttt{i}
    \item \texttt{function set (a: array 't) (i: int) (v: 't): array 't}
    -- изменение элемента с индексом \texttt{i}
    \end{itemize}
    \end{frame}

    \begin{frame}[fragile]{Аксиомы}
    \begin{lstlisting}
    axiom get_set: forall a i j v.
        get (set a i v) j =
        (if i = j then v else get a j)
    axiom len_set: forall a i v.
        len (set a i v) = len a
    axiom len_nonneg: forall a.
        len a >= 0
    \end{lstlisting}
    \end{frame}

    \begin{frame}[fragile]{Примитивы для операторов CALL}
    \begin{lstlisting}
val ln (a: arrayVar 't): int
    ensures { result = len a.items }
val gt (a: arrayVar 't) (i: int) : 't
    requires { valid_index a.items i }
    ensures { result = get a.items i }
val st (a: arrayVar 't) (i: int) (v: 't): array 't
    requires { valid_index a.items i }
    ensures { result = set a.items i }
val make (sz: int) (v: 't): array 't
    requires { sz >= 0 }
    ensures { len result = sz }
    ensures { forall i. 0 <= i < sz ->
        get result i = v }
    \end{lstlisting}
    \end{frame}

    \begin{frame}{Пример}
    \begin{itemize}
    \item
    На языке WhyML напишите модель программы,
    решающей следующую задачу: на вход подается
    массив и два значения; нужно заменить в массиве все
    вхождения первого значения на второе.
    В этой модели входные переменные не должны
    изменяться.
    \item
    Напишите спецификацию задачи на языке Why.
    \item
    При помощи методов Флойда докажите полную
    корректность модели программы относительно
    этой спецификации.
    \end{itemize}
    \end{frame}

    \begin{frame}{Конструкции old, at. Метки}
    \begin{itemize}
    \item
    Разрешим изменять входные переменные, т.е.
    сделаем типом входной переменной тип \texttt{arrayVar}.
    \item
    Теперь в спецификациях нужно обращаться
    к значению переменной-массива, которое было
    у нее при вызове функции. Для этого нужны следующие конструкции:
    \item
    \texttt{old expr} -- это для постусловия
    \item
    \texttt{at expr 'label} -- это для спецификаций внутри функции,
    надо создать метку в коде функции (\texttt{'label: expr})
    \end{itemize}
    \end{frame}

\end{document}
